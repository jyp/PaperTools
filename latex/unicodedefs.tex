\DeclareUnicodeCharacter{00A0}{~} %   NO-BREAK SPACE
\DeclareUnicodeCharacter{00A7}{\S} % §
\DeclareUnicodeCharacter{00AC}{\ensuremath{\neg}} % ¬
\DeclareUnicodeCharacter{00B0}{^{\circ}} % °
\DeclareUnicodeCharacter{00B1}{^1} 
\DeclareUnicodeCharacter{00B2}{^2} % ²
\DeclareUnicodeCharacter{00B7}{\ensuremath{\cdot}} % ·
\DeclareUnicodeCharacter{00B9}{\textsuperscript{l}} % ¹
\DeclareUnicodeCharacter{00D7}{\ensuremath{\times}} % × 
\DeclareUnicodeCharacter{00F7}{\ensuremath{\div}} % ÷
\DeclareUnicodeCharacter{02B3}{\ensuremath{{^r}}} % ʳ
\DeclareUnicodeCharacter{02E1}{\ensuremath{{^l}}} % ˡ
\DeclareUnicodeCharacter{0393}{\ensuremath{\Gamma}} % Γ
\DeclareUnicodeCharacter{0394}{\ensuremath{\Delta}} % Δ
\DeclareUnicodeCharacter{0397}{\ensuremath{\textrm{H}}} % Η
\DeclareUnicodeCharacter{0398}{\ensuremath{\Theta}} % Θ
\DeclareUnicodeCharacter{039B}{\ensuremath{\Lambda}} % Λ
\DeclareUnicodeCharacter{039E}{\ensuremath{\Xi}} % Ξ
\DeclareUnicodeCharacter{03A3}{\ensuremath{\Sigma}} % Σ
\DeclareUnicodeCharacter{03A6}{\ensuremath{\Phi}} % Φ
\DeclareUnicodeCharacter{03A8}{\ensuremath{\Psi}} % Ψ
\DeclareUnicodeCharacter{03A9}{\ensuremath{\Omega}} % Ω
\DeclareUnicodeCharacter{03B1}{\ensuremath{\mathnormal{\alpha}}} % α
\DeclareUnicodeCharacter{03B2}{\ensuremath{\beta}} % β
\DeclareUnicodeCharacter{03B3}{\ensuremath{\mathnormal{\gamma}}} % γ
\DeclareUnicodeCharacter{03B4}{\ensuremath{\mathnormal{\delta}}} % δ
\DeclareUnicodeCharacter{03B5}{\ensuremath{\mathnormal{\varepsilon}}} % ε
\DeclareUnicodeCharacter{03B6}{\ensuremath{\mathnormal{\zeta}}} % ζ
\DeclareUnicodeCharacter{03B7}{\ensuremath{\mathnormal{\eta}}} % η
\DeclareUnicodeCharacter{03B8}{\ensuremath{\mathnormal{\theta}}} % θ
\DeclareUnicodeCharacter{03B9}{\ensuremath{\mathnormal{\iota}}} % ι
\DeclareUnicodeCharacter{03BA}{\ensuremath{\mathnormal{\kappa}}} % κ
\DeclareUnicodeCharacter{03BB}{\ensuremath{\mathnormal{\lambda}}} % λ
\DeclareUnicodeCharacter{03BC}{\ensuremath{\mathnormal{\mu}}} % μ
\DeclareUnicodeCharacter{03BD}{\ensuremath{\mathnormal{\mu}}} % ν
\DeclareUnicodeCharacter{03BE}{\ensuremath{\mathnormal{\xi}}} % ξ
\DeclareUnicodeCharacter{03C0}{\ensuremath{\mathnormal{\pi}}} % π
\DeclareUnicodeCharacter{03C1}{\ensuremath{\mathnormal{\rho}}} % ρ
\DeclareUnicodeCharacter{03C3}{\ensuremath{\mathnormal{\sigma}}} % σ
\DeclareUnicodeCharacter{03C4}{\ensuremath{\mathnormal{\tau}}} % τ
\DeclareUnicodeCharacter{03C6}{\ensuremath{\mathnormal{\varphi}}} % φ
\DeclareUnicodeCharacter{03C7}{\ensuremath{\mathnormal{\chi}}} % χ
\DeclareUnicodeCharacter{03C8}{\ensuremath{\mathnormal{\psi}}} % ψ
\DeclareUnicodeCharacter{03C9}{\ensuremath{\mathnormal{\omega}}} % ω 
\DeclareUnicodeCharacter{03D5}{\ensuremath{\mathnormal{\phi}}} % ϕ
\DeclareUnicodeCharacter{03F5}{\ensuremath{\mathnormal{\epsilon}}} % ϵ
\DeclareUnicodeCharacter{10627}{\ensuremath{\lbana}} 
\DeclareUnicodeCharacter{10628}{\ensuremath{\rbana}} 
\DeclareUnicodeCharacter{1D62}{_i} % ᵢ
\DeclareUnicodeCharacter{2026}{\ensuremath{\ldots}}
\DeclareUnicodeCharacter{202F}{{\,}}
\DeclareUnicodeCharacter{2032}{\ensuremath{^\prime}}  % ′ 
\DeclareUnicodeCharacter{2033}{\ensuremath{^\second}} % ″ 
\DeclareUnicodeCharacter{2034}{\ensuremath{^\third}}  % ‴ 
\DeclareUnicodeCharacter{2080}{\ensuremath{_0}} % ₀
\DeclareUnicodeCharacter{2081}{\ensuremath{_1}}
\DeclareUnicodeCharacter{2082}{\ensuremath{_2}}
\DeclareUnicodeCharacter{2083}{\ensuremath{_3}}
\DeclareUnicodeCharacter{2084}{\ensuremath{_4}}
\DeclareUnicodeCharacter{2085}{\ensuremath{_5}}
\DeclareUnicodeCharacter{2086}{\ensuremath{_6}}
\DeclareUnicodeCharacter{2087}{\ensuremath{_7}}
\DeclareUnicodeCharacter{2088}{\ensuremath{_8}}
\DeclareUnicodeCharacter{2089}{\ensuremath{_9}}
\DeclareUnicodeCharacter{2102}{\ensuremath{\mathbb{C}}} % ℂ 
\DeclareUnicodeCharacter{2115}{\ensuremath{\mathbb{N}}} % 
\DeclareUnicodeCharacter{211D}{\ensuremath{\mathbb{R}}} % ℝ 
\DeclareUnicodeCharacter{2124}{\ensuremath{\mathbb{Z}}} % ℤ 
\DeclareUnicodeCharacter{214B}{\ensuremath{\parr}} % ⅋
\DeclareUnicodeCharacter{2190}{\ensuremath{\leftarrow}} % ← 
\DeclareUnicodeCharacter{2191}{\ensuremath{\uparrow}} % ↑
\DeclareUnicodeCharacter{2192}{\ensuremath{\rightarrow}} % →
\DeclareUnicodeCharacter{2193}{\ensuremath{\downarrow}} % ↓
\DeclareUnicodeCharacter{2194}{\ensuremath{\leftrightarrow}} % ↔
\DeclareUnicodeCharacter{2196}{\nwarrow} % ↖
\DeclareUnicodeCharacter{2197}{\nearrow} % ↗
\DeclareUnicodeCharacter{219D}{\ensuremath{\leadsto}} % ↝
\DeclareUnicodeCharacter{21A6}{\ensuremath{\mapsto}} % ↦ 
\DeclareUnicodeCharacter{21C6}{\ensuremath{\leftrightarrows}} % ⇆
\DeclareUnicodeCharacter{21D0}{\ensuremath{\Leftarrow}} % ⇐
\DeclareUnicodeCharacter{21D2}{\ensuremath{\Rightarrow}} % ⇒ 
\DeclareUnicodeCharacter{21D4}{\ensuremath{\Leftrightarrow}} % ⇔
\DeclareUnicodeCharacter{2200}{\ensuremath{\forall}} % ∀
\DeclareUnicodeCharacter{2203}{\ensuremath{\exists}} % ∃
\DeclareUnicodeCharacter{2205}{\ensuremath{\varnothing}} % ∅
\DeclareUnicodeCharacter{2208}{\ensuremath{\in}} % ∈
\DeclareUnicodeCharacter{2209}{\ensuremath{\not\in}} % ∉
\DeclareUnicodeCharacter{220B}{\ensuremath{\ni}}
\DeclareUnicodeCharacter{220E}{\ensuremath{\qed}} % ∎ % Alternatively use \blacksquare
\DeclareUnicodeCharacter{2211}{\sum}% ∑
\DeclareUnicodeCharacter{2215}{\mathbb{N}} % ℕ
\DeclareUnicodeCharacter{2217}{\ensuremath{\ast}} % ∗
\DeclareUnicodeCharacter{2218}{\ensuremath{\circ}} % ∘
\DeclareUnicodeCharacter{2219}{\ensuremath{\bullet}} % ∙ 
\DeclareUnicodeCharacter{221E}{\ensuremath{\infty}} % ∞
\DeclareUnicodeCharacter{2223}{\ensuremath{\mid}} % ∣
\DeclareUnicodeCharacter{2227}{\wedge}% ∧
\DeclareUnicodeCharacter{2228}{\vee}% ∨
\DeclareUnicodeCharacter{2229}{\ensuremath{\cap}} % ∩
\DeclareUnicodeCharacter{222A}{\ensuremath{\cup}} % ∪
\DeclareUnicodeCharacter{2237}{::} % ∷
\DeclareUnicodeCharacter{223C}{\ensuremath{\sim}} % ∼
\DeclareUnicodeCharacter{2243}{\ensuremath{\simeq}} % ≃
\DeclareUnicodeCharacter{2245}{\ensuremath{\cong}} % ≅ 
\DeclareUnicodeCharacter{2248}{\ensuremath{\approx}} % ≈
\DeclareUnicodeCharacter{225C}{\ensuremath{\stackrel{\scriptscriptstyle {\triangle}}{=}}} % ≜
\DeclareUnicodeCharacter{225F}{\ensuremath{\stackrel{\scriptscriptstyle ?}{=}}} % ≟
\DeclareUnicodeCharacter{2260}{\neq}% ≠
\DeclareUnicodeCharacter{2261}{\ensuremath{\equiv}}% ≡
\DeclareUnicodeCharacter{2264}{\ensuremath{\le}} % ≤
\DeclareUnicodeCharacter{2265}{\ensuremath{\ge}} % ≥
\DeclareUnicodeCharacter{2282}{\ensuremath{\subset}} % ⊂
\DeclareUnicodeCharacter{2283}{\ensuremath{\supset}} % ⊃ 
\DeclareUnicodeCharacter{2286}{\ensuremath{\subseteq}} % ⊆ 
\DeclareUnicodeCharacter{2287}{\ensuremath{\supseteq}} % ⊇
\DeclareUnicodeCharacter{2293}{\ensuremath{\sqcup}} % ⊓
\DeclareUnicodeCharacter{2293}{\sqcap} % ⊓
\DeclareUnicodeCharacter{2294}{\sqcup} % ⊔
\DeclareUnicodeCharacter{2295}{\ensuremath{\oplus}} % ⊕
\DeclareUnicodeCharacter{2297}{\ensuremath{\otimes}} % ⊗
\DeclareUnicodeCharacter{22A2}{\ensuremath{\vdash}}
\DeclareUnicodeCharacter{22A4}{\ensuremath{\top}} % ⊤
\DeclareUnicodeCharacter{22A5}{\ensuremath{\bot}} % ⊥
\DeclareUnicodeCharacter{22A7}{\models} % ⊧ 
\DeclareUnicodeCharacter{22A8}{\models} % ⊨
\DeclareUnicodeCharacter{22A9}{\Vdash} % ⊩
\DeclareUnicodeCharacter{22B8}{\ensuremath{\multimap}} % ⊸
\DeclareUnicodeCharacter{22C4}{\ensuremath{\diamond}} % ⋄
\DeclareUnicodeCharacter{22C6}{\ensuremath{\star}}
\DeclareUnicodeCharacter{22EE}{\ensuremath{\vdots}} % ⋮
\DeclareUnicodeCharacter{22EF}{\ensuremath{\cdots}} % ⋯
\DeclareUnicodeCharacter{2308}{\ensuremath{\lceil}}
\DeclareUnicodeCharacter{2309}{\ensuremath{\rceil}}
\DeclareUnicodeCharacter{230A}{\ensuremath{\lfloor}}
\DeclareUnicodeCharacter{230B}{\ensuremath{\rfloor}}
\DeclareUnicodeCharacter{25A1}{\ensuremath{\square}} % □
\DeclareUnicodeCharacter{25AF}{\mathop{\talloblong}} % ▯
\DeclareUnicodeCharacter{25B9}{\ensuremath{\rhd}} % ▹
\DeclareUnicodeCharacter{25C7}{\ensuremath{\diamond}} % ◇
\DeclareUnicodeCharacter{2605}{\ensuremath{\star}}   % ★
\DeclareUnicodeCharacter{2713}{\ensuremath{\checkmark}} % ✓
\DeclareUnicodeCharacter{27C2}{\ensuremath{^\bot}} % PERPENDICULAR ⟂
\DeclareUnicodeCharacter{27E6}{\ensuremath{\llbracket}} % ⟦
\DeclareUnicodeCharacter{27E7}{\ensuremath{\rrbracket}} % ⟧
\DeclareUnicodeCharacter{27E8}{\ensuremath{\langle}} % ⟨
\DeclareUnicodeCharacter{27E9}{\ensuremath{\rangle}} % ⟩
\DeclareUnicodeCharacter{27F6}{{\longrightarrow}} % ⟶
\DeclareUnicodeCharacter{27F7}{{\longleftrightarrow}} % ⟷
\DeclareUnicodeCharacter{2A02}{\ensuremath{\bigotimes}} % ⨂
\DeclareUnicodeCharacter{2A04}{\mathop{\dot{\cup}}} % ⨄
\DeclareUnicodeCharacter{2AFE}{\mathop{\talloblong}} % ⫾5

% \DeclareUnicodeCharacter{8499}{\mathcal{M}} 
% \DeclareUnicodeCharacter{8718}{\ensuremath{\blacksquare}}
% \DeclareUnicodeCharacter{8797}{\mathrel{\mathop:}=}
% \DeclareUnicodeCharacter{9657}{\ensuremath{\triangleright}}
% \DeclareUnicodeCharacter{9667}{\triangleright{}}
% \DeclareUnicodeCharacter{9669}{\ensuremath{\triangleleft}}
